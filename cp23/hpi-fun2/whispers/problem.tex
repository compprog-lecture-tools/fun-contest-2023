\input{template.tex}

\begin{document}

\makeheader

Whispers down the line: Certification of a Postal Company

In a world dominated by digital communication and instant message delivery, a postal company aspires to achieve something extraordinary - the proven and reliable transmission of information in a traditional manner. 
Instead of relying on electronic letters, this company places its trust in the oral transmission of messages through dedicated postal employees. 
A message is passed on to one employee who then passes it to another, until finally, a responsible employee delivers the message to the intended recipient. 
In those information chains the employees communicate exclusively with their direct superiors or subordinates.

However, the paths of communication are not without challenges. 
The postal company has recognized that due to language barriers and miscommunication each step of the transmission process carries a certain level of uncertainty. 
Between the individual links of the information chain, there is always a certain amount of misinformation added to the message.

In order to maintain the trust of its customers, the company has decided to subject its postal system to a rigorous certification process.
The certification agency provides a value of acceptable misinformation with which it is still possible to reconstruct the original message.
The certification is successful, if between any two employees the amount of misinformation added up during the transmitting of the message is not higher than the given value.

The postal company now seeks your assistance in determining whether it can attain the coveted certification or if it must confront the realities of modern communication.


\paragraph*{Input}

In the first line of input you get the value of acceptable misinformation $0 \leq c \leq 10^{10}$ and the number of employees $0 \leq n \leq 2\cdot 10^5.$
In the following $n-1$ lines you get the two employees $0 \leq a_i, b_i < n$ and their amount of miscommunication $0 \leq c_i \leq 10^{10}.$

\paragraph*{Output}

Print "Yes: ", if the certification was successful, and "No: ", if not, followed by the maximum amount of misinformation added to a message during the transfer between any two employees.

\begin{samples}
  \sample{sample1}
  \sample{sample2}
\end{samples}

\end{document}
