\input{template.tex}

\begin{document}

\makeheader

\placeholder{Problem description} 
\\
In 2023 Hasso Plattner is launching his own currency (HPC). In the basement of the newly build buildings at Campus 2 the money press will be guarded by a complex alarm system. 
You want to be able to print your own HPC and therefore you need the money printing plates from the basement. 
In the basement is a long hallway of dimensions (width 10m and length 100m) separates you from money press. There are n motion detectors on the ceiling that detect intruders if they pass within a circular range of r. 
Your goal is to find out, if it is possible to reach the other end of the hallway without triggering the alarm. 

\paragraph*{Input}
\placeholder{Input description}
n: an integer that describes the number of motion detectors \\
followed by n lines of:
$x_i$, $y_i$ and $r_i$: the position and the range of the ith motion detector


\paragraph*{Output}
\placeholder{Output description}
“Yes” if it is possible to reach the other end of the hallway \\
“No” if it is not possible to reach the other end of the hallway


\begin{samples}
  \sample{sample1}
  \begin{verbatim}
    50 10 2.5
    50 5 2.5
    50 0 2.5 
    
    No   
  \end{verbatim}
    

    
  \sample{sample2}
  \begin{verbatim}
    50 10 2.5
    40 5 2.5
    50 0 2.5 
    
    Yes  
  \end{verbatim}
\end{samples}

\placeholder{Solution sketch}
\\
Every motion detector is a node in a graph, furthermore there is a node $n_upper$ and $n_lower$. 
An alarm node $n_i$ is connected to another alarm node $n_j$ if they are at most $r_i$ + $r_j$ apart from each other. 
$n_upper$ is connected to each alarm node $n_i$ that is at most $r_i$ apart from upper border, the same hold for $n_lower$ with respect to the lower border. 
On that graph we execute a dfs or bfs with $n_upper$ as the start node. If $n_lower$ is reachable, there is a line of motion detectors that spans from top to bottom, thus making it impossible to reach the other side of the room.

\end{document}