\input{template.tex}

\begin{document}

\makeheader

After graduating from HPI, your friend started a phone tech-support
business. He wants to offer his customers a 24/7 available, in most cases quite helpful
service via a phone hotline. Surely it is impossible for him to run this
business alone: he already started to employ a whole lot of call agents, which
are set to communicate the very best answers to the questions his customers call the company
for.

However, he does not have a phone number yet. Because all of his employees need
individual call-throughs to call each other during breaks, he applies at the phone company for
an entire block of consecutive landline numbers. By the way, he also owns a nice
electric vehicle, on which he wants to put the company main hotline on. There is
just one small problem: he only has one sticker for each decimal digit. Is it
even possible for him to find a phone number in an offered number block which he can
stick onto his car?

\paragraph*{Input}

You are given two numbers $1\leq l \leq r\leq 10^6$, which represent the first
and last call-through's in an offered block.

\paragraph*{Output}

Print a number $l\leq x\leq r$ inside the block, which he can put as company branding on
his car with the stickers. If this is not possible, print \texttt{-1}. If there are multiple solutions, print any of them.

\begin{samples}
  \sample{sample1}
  \sample{sample2}
\end{samples}

\end{document}
