\input{template.tex}

\begin{document}

\makeheader
You and your rattlegang are hanging out at the park at night. 
One of your fellow rattlers starts talking about a cool partner-rattling-workshop taking place next weekend. Of course the whole gang wants to participate and learn about improving their skills on the rattle, especially since the workshop focuses on rattling in sync with your rattle-partner. 
Of course partner-rattling is quite complicated and requires a whole lot of trust and mutual understanding of rhythm and music, so every one of you can only imagine signing up for the workshop with a close friend and mutual rattler. 
However, such a rattlegang is a quite complex and rigid social network. Every gang member considers themselves either a fan of fast glasses or a mullet maniac. To never hang out with a friend and having to worry about them having a faster pair of glasses or a more meticulously styled mullet every fan of fast glasses is only friends with mullet maniacs and vice versa. Since a rattle gang needs to be able to operate at any time and without any jealousy every gang member is only permitted to have exactly 3 friends within the gang. You are wondering what's the largest number of pairs of friends who can sign up for the partner-rattling-workshop. 


\paragraph*{Input}
The first line contains $n$ ($6 \leq n \leq $ tbd), the number of members of the rattle gang. The following lines contain all friendships within your gang, denoted by two numbers $a$ and $b$ ($1 \leq a,b \leq n, a \neq b$), indicating that the gang members $a$ and $b$ are friends. % No m necessary since the degree of the nodes determines the number of edges, also no need to even read the edges

\paragraph*{Output}
Output a single integer: The largest number of pairs of friends that you can form within your gang in order to participate in the workshop. 

\begin{samples}
  \sample{sample1}
  \sample{sample2}
  \sample{sample3}
\end{samples}

\end{document}