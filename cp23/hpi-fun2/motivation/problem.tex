\input{template.tex}

\begin{document}

\makeheader

One day, Christian, the CEO of \textit{Super Arrangement Productions}, entered one of their offices
to talk to an employee who complained about the lack of motivation in their company. 
Christian was quite surprised, that not only the department, but the whole company was not motivated
at all.

That is why Christian made the decision to hire a motivation agency to send the best of their coaches
to help motivate the people again.

In a brief knowledge exchange with the head of the coaches, Christian learned the following things:

\begin{itemize}
  \item Employees have to be motivated by coaches individually, but an employee will not allow getting motivated
        by a coach that their boss did not get motivated by.
  \item Employees are as motivated as the coach motivating them, after they got coached.
  \item Each employee has a demotivational level, which tells how much effort it needs for a coach to
        motivate the employee again.
  \item After being motivated again, employees are willing to motivate other employees, but only if those
        employees are directly or indirectly under their current position.
  \item Motivating employees is exhausting, which means that coaches and employees motivating other employees
        will be less motivated, equally to the level of demotivation.
  \item The agency sends their best coaches with each having a motivational level of 20.
\end{itemize}

Since Christian enforced saving measures, he wants to spend as little as possible. Therfore, as few coaches 
as possible should be sent. They will start working with Christian himself.

Even though the whole company should become motivated again, some people simply might not be motivatable.
Christian wants to know who these people are.

\paragraph*{Input}

The input consists of three lines, describing a single test case.
The first line contains an integer $n$ ($1 \leq n \leq 10^5$), the number of employees.
The second line contains $n$ integers. The $i$th integer is $mi$ ($0 \leq mi \leq 20$), the motivational level of employee $i$.
The third line contains $n - 1$ integers. The $i$th integer is $bi+1$ ($1 \leq bi+1 < i + 1$), the direct boss of colleague $i + 1$.

\paragraph*{Output}

Firstly, output the minimal number of coaches required to motivate the company. Secondly, output the number of 
unmotivatable employees, followed by the numbers of those employees.


\begin{samples}
  \sample1
  \begin{verbatim}
    11
    2 5 3 7 2 1 10 4 1 8 2
    1 2 2 2 5 5 1 8 8 8

    3
    0
    ~
  \end{verbatim}
\end{samples}

\end{document}
